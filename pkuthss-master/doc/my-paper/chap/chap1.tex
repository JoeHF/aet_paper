% Copyright (c) 2014,2016 Casper Ti. Vector
% Public domain.

\chapter{动机}
\section{MRC算法}
失效率曲线(MRC)是衡量内存大小和性能之间的一个重要的工具,它刻画了不同缓存大小下所对应的失效率,通过MRC,我们可以重新定义工作集大小,即在不显著影响应用程序性能条件下所需要的缓存大小。

如图所示,当缓存大小为0的时候,失效率为100\%,即所有的访问在缓存中均会失效,随着缓存大小的增大,失效率将会逐渐降低,这是由程序的局部性决定的,这里的局部性主要指的是时间局部性(如果一个信息项正在被访问,那么在近期它很可能还会再次被访问)。传统的计算MRC的算法是LRU栈算法\supercite{Mattson1970Evaluation},


% vim:ts=4:sw=4
